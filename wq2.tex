\documentclass[final,t]{beamer}\usepackage[]{graphicx}\usepackage[]{color}
% maxwidth is the original width if it is less than linewidth
% otherwise use linewidth (to make sure the graphics do not exceed the margin)
\makeatletter
\def\maxwidth{ %
  \ifdim\Gin@nat@width>\linewidth
    \linewidth
  \else
    \Gin@nat@width
  \fi
}
\makeatother

\definecolor{fgcolor}{rgb}{0.345, 0.345, 0.345}
\newcommand{\hlnum}[1]{\textcolor[rgb]{0.686,0.059,0.569}{#1}}%
\newcommand{\hlstr}[1]{\textcolor[rgb]{0.192,0.494,0.8}{#1}}%
\newcommand{\hlcom}[1]{\textcolor[rgb]{0.678,0.584,0.686}{\textit{#1}}}%
\newcommand{\hlopt}[1]{\textcolor[rgb]{0,0,0}{#1}}%
\newcommand{\hlstd}[1]{\textcolor[rgb]{0.345,0.345,0.345}{#1}}%
\newcommand{\hlkwa}[1]{\textcolor[rgb]{0.161,0.373,0.58}{\textbf{#1}}}%
\newcommand{\hlkwb}[1]{\textcolor[rgb]{0.69,0.353,0.396}{#1}}%
\newcommand{\hlkwc}[1]{\textcolor[rgb]{0.333,0.667,0.333}{#1}}%
\newcommand{\hlkwd}[1]{\textcolor[rgb]{0.737,0.353,0.396}{\textbf{#1}}}%
\let\hlipl\hlkwb

\usepackage{framed}
\makeatletter
\newenvironment{kframe}{%
 \def\at@end@of@kframe{}%
 \ifinner\ifhmode%
  \def\at@end@of@kframe{\end{minipage}}%
  \begin{minipage}{\columnwidth}%
 \fi\fi%
 \def\FrameCommand##1{\hskip\@totalleftmargin \hskip-\fboxsep
 \colorbox{shadecolor}{##1}\hskip-\fboxsep
     % There is no \\@totalrightmargin, so:
     \hskip-\linewidth \hskip-\@totalleftmargin \hskip\columnwidth}%
 \MakeFramed {\advance\hsize-\width
   \@totalleftmargin\z@ \linewidth\hsize
   \@setminipage}}%
 {\par\unskip\endMakeFramed%
 \at@end@of@kframe}
\makeatother

\definecolor{shadecolor}{rgb}{.97, .97, .97}
\definecolor{messagecolor}{rgb}{0, 0, 0}
\definecolor{warningcolor}{rgb}{1, 0, 1}
\definecolor{errorcolor}{rgb}{1, 0, 0}
\newenvironment{knitrout}{}{} % an empty environment to be redefined in TeX

\usepackage{alltt}
\mode<presentation>{\usetheme{I6dv}}

% settings
\setbeamerfont{itemize}{size=\normalsize}
\setbeamerfont{itemize/enumerate body}{size=\normalsize}
\setbeamerfont{itemize/enumerate subbody}{size=\normalsize}
\setbeamertemplate{caption}[numbered]

% packages
\usepackage{xcolor}
\usepackage{times}
\usepackage{amsmath,amsthm, amssymb, latexsym}
\usepackage{exscale}
\usepackage{subfig}
\usepackage{booktabs, array}
\usepackage{tabularx}
\usepackage[english]{babel}
\usepackage[latin1]{inputenc}
\usepackage[orientation=landscape,size=custom,width=21.59,height=27.94,scale=0.45]{beamerposter} % in cm, equal to 8.5" wide x 11" high
\usepackage{color, colortbl}
% \usepackage[printwatermark]{xwatermark}
\usepackage{graphicx}
\usepackage{tikz}

% % for watermark using tikz
% \newsavebox\mybox
% \savebox\mybox{\tikz[color=red,opacity=0.3]\node{DRAFT};}
% \newwatermark*[
%   allpages,
%   angle=45,
%   scale=6,
%   xpos=-20,
%   ypos=15
% ]{\usebox\mybox}

\setcounter{figure}{4}

\title{\Large Progress Towards Meeting Regulatory Goals}
\author{\normalsize An Initiative of the Tampa Bay Nitrogen Management Consortium to Maintain\\ and Restore the Bay's Resources}



\IfFileExists{upquote.sty}{\usepackage{upquote}}{}
\begin{document}

\begin{frame}

\vspace{-0.4cm} %spacing for block distance from header
\begin{columns}[t]

%%%%%%%%%%%%%%
% left
%%%%%%%%%%%%%%
\begin{column}{.2\linewidth}

\vspace{-0.2in}


\begin{figure}
\centerline{\includegraphics[trim = 0cm 0cm 0cm 0cm, width=1.1\linewidth]{figure/chlmat.pdf}}
\caption{\footnotesize Attainment of bay segments for chlorophyll criteria from 1975 to 2020 (April, May data missing for 2020).}
\label{fig:chlmat}
\end{figure}

\end{column}

%%%%%%%%%%%%%%
% right
%%%%%%%%%%%%%%    
\begin{column}{.79\linewidth}

\begin{block}{Maintaining Reasonable Assurance \& TMDL Compliance}
\footnotesize During 2020, the COVID-19 pandemic precluded water quality data collection in April and May. As a result, compliance determinations have not been made for any bay segments. Results shown in Figure \ref{fig:chlmat} depict chlorophyll-a concentrations in relation to regulatory criteria, as calculated without observations from the months noted above. The fourth RA annual assessment report for the 2017-2021 period will be submitted in April 2021.
\end{block}

\begin{block}{2020 Chl-a Monthly Variation Compared to 1974-2019}
\footnotesize
Chlorophyll-a concentrations were evaluated within the bay on a monthly basis during 2020 and compared to prior years' levels (Figure \ref{fig:chlboxplot}). Elevated concentrations in Old Tampa Bay were primarily due to \textit{Pyrodinium bahamense} during the late summer months.
\end{block}



\vspace{-0.2in}

\begin{figure}
\centerline{\includegraphics[trim = 0cm 0cm 0cm 0cm, width=1\linewidth]{figure/chlboxplot.pdf}}
\caption{\footnotesize Chlorophyll-a monthly averages from 1975-2019 for the four bay segments. The monthly averages for 2020 are shown in red.}
\label{fig:chlboxplot}
\end{figure}

\vspace{-0.4in}

\begin{block}{Tampa Bay Seagrass Recovery}
\vspace{-0.15in}
\begin{minipage}{0.6\textwidth}
\footnotesize
Tampa Bay's total seagrass coverage remains above the recovery goal, though a slight decrease was observed from 2016 to 2018. The 2018 baywide coverage was estimated at 40,652 acres (Figure \ref{fig:sgtrnd}). As in 2016, coverage remains above the target (40,000 acres) and the estimated historic coverage of the 1950s (40,420 acres). SWFWMD coverage estimates from the winter 2019-20 period will be available in spring 2021. More information on assessments of the bay's seagrass recovery using transect monitoring data can be found at \href{https://shiny.tbep.org/seagrasstransect-dash/}{https://shiny.tbep.org/seagrasstransect-dash/} and using the coverage estimates from SWFWMD can be found at \href{https://shiny.tbep.org/seagrasscoverage-dash/}{https://shiny.tbep.org/seagrasscoverage-dash/}.
\end{minipage}
\hspace{0.2in}
\begin{minipage}{0.32\textwidth}
\vspace{0.1in}
\begin{figure}
\includegraphics[width=\textwidth, trim = 0cm 0cm 0cm -1cm]{www/Seagrass_Acreage_1950_2018.png}
\caption{\footnotesize Seagrass estimates from 1950-2018 (Source: TBEP \& SWFWMD)}
\label{fig:sgtrnd}
\end{figure}
\end{minipage}
\end{block}

\vspace{-0.1in}

\tiny \textit{\textbf{Note}: 2020 nutrient management compliance assessment available from Sherwood, E., Burke, M., Beck, M.W. 2020. TBEP Technical Report \#06-21.  Please cite this document as Beck, M.W., Burke, M., Raulerson, G. 2021. 2020 Tampa Bay Water Quality Assessment. TBEP Technical Report \#05-21, St. Petersburg, FL.} \\

\end{column}

\end{columns}

\end{frame}

\end{document}
