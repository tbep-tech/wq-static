\documentclass[final,t]{beamer}\usepackage[]{graphicx}\usepackage[]{color}
% maxwidth is the original width if it is less than linewidth
% otherwise use linewidth (to make sure the graphics do not exceed the margin)
\makeatletter
\def\maxwidth{ %
  \ifdim\Gin@nat@width>\linewidth
    \linewidth
  \else
    \Gin@nat@width
  \fi
}
\makeatother

\definecolor{fgcolor}{rgb}{0.345, 0.345, 0.345}
\newcommand{\hlnum}[1]{\textcolor[rgb]{0.686,0.059,0.569}{#1}}%
\newcommand{\hlstr}[1]{\textcolor[rgb]{0.192,0.494,0.8}{#1}}%
\newcommand{\hlcom}[1]{\textcolor[rgb]{0.678,0.584,0.686}{\textit{#1}}}%
\newcommand{\hlopt}[1]{\textcolor[rgb]{0,0,0}{#1}}%
\newcommand{\hlstd}[1]{\textcolor[rgb]{0.345,0.345,0.345}{#1}}%
\newcommand{\hlkwa}[1]{\textcolor[rgb]{0.161,0.373,0.58}{\textbf{#1}}}%
\newcommand{\hlkwb}[1]{\textcolor[rgb]{0.69,0.353,0.396}{#1}}%
\newcommand{\hlkwc}[1]{\textcolor[rgb]{0.333,0.667,0.333}{#1}}%
\newcommand{\hlkwd}[1]{\textcolor[rgb]{0.737,0.353,0.396}{\textbf{#1}}}%
\let\hlipl\hlkwb

\usepackage{framed}
\makeatletter
\newenvironment{kframe}{%
 \def\at@end@of@kframe{}%
 \ifinner\ifhmode%
  \def\at@end@of@kframe{\end{minipage}}%
  \begin{minipage}{\columnwidth}%
 \fi\fi%
 \def\FrameCommand##1{\hskip\@totalleftmargin \hskip-\fboxsep
 \colorbox{shadecolor}{##1}\hskip-\fboxsep
     % There is no \\@totalrightmargin, so:
     \hskip-\linewidth \hskip-\@totalleftmargin \hskip\columnwidth}%
 \MakeFramed {\advance\hsize-\width
   \@totalleftmargin\z@ \linewidth\hsize
   \@setminipage}}%
 {\par\unskip\endMakeFramed%
 \at@end@of@kframe}
\makeatother

\definecolor{shadecolor}{rgb}{.97, .97, .97}
\definecolor{messagecolor}{rgb}{0, 0, 0}
\definecolor{warningcolor}{rgb}{1, 0, 1}
\definecolor{errorcolor}{rgb}{1, 0, 0}
\newenvironment{knitrout}{}{} % an empty environment to be redefined in TeX

\usepackage{alltt}
\mode<presentation>{\usetheme{I6dv}}

% settings
\setbeamerfont{itemize}{size=\normalsize}
\setbeamerfont{itemize/enumerate body}{size=\normalsize}
\setbeamerfont{itemize/enumerate subbody}{size=\normalsize}
\setbeamertemplate{caption}[numbered]

% packages
\usepackage{xcolor}
\usepackage{times}
\usepackage{amsmath,amsthm, amssymb, latexsym}
\usepackage{exscale}
\usepackage{subfig}
\usepackage{booktabs, array}
\usepackage{tabularx}
\usepackage[english]{babel}
\usepackage[latin1]{inputenc}
\usepackage[orientation=landscape,size=custom,width=21.59,height=27.94,scale=0.45]{beamerposter} % in cm, equal to 8.5" wide x 11" high
\usepackage{color, colortbl}
% \usepackage[printwatermark]{xwatermark}
\usepackage{graphicx}
\usepackage{tikz}

% color links
\definecolor{links}{HTML}{2A1B81}
\hypersetup{colorlinks,linkcolor=,urlcolor=links}

% % for watermark using tikz
% \newsavebox\mybox
% \savebox\mybox{\tikz[color=red,opacity=0.3]\node{DRAFT};}
% \newwatermark*[
%   allpages,
%   angle=45,
%   scale=6,
%   xpos=-20,
%   ypos=15
% ]{\usebox\mybox}

\setcounter{figure}{4}

\title{\Large Progress Towards Meeting Regulatory Goals}
\author{\normalsize An Initiative of the Tampa Bay Nitrogen Management Consortium to Maintain\\ and Restore the Bay's Resources}



\IfFileExists{upquote.sty}{\usepackage{upquote}}{}
\begin{document}

\begin{frame}

\vspace{-0.4cm} %spacing for block distance from header
\begin{columns}[t]

%%%%%%%%%%%%%%
% left
%%%%%%%%%%%%%%
\begin{column}{.2\linewidth}

\vspace{-0.2in}


\begin{figure}
\centerline{\includegraphics[trim = 0cm 0cm 0cm 0cm, width=1.1\linewidth]{figure/chlmat.pdf}}
\caption{\footnotesize Bay segment attainment of chlorophyll criteria from 1975 to 2021 (April, May data missing for 2020).}
\label{fig:chlmat}
\end{figure}

\end{column}

%%%%%%%%%%%%%%
% right
%%%%%%%%%%%%%%    
\begin{column}{.79\linewidth}

\begin{block}{Maintaining Reasonable Assurance \& TMDL Compliance}
\footnotesize In 2021, all bay segments met FDEP criteria for chlorophyll, except Old Tampa Bay. The criteria was exceeded for the third year in a row, requiring additional actions by the Tampa Bay Nitrogen Management Consortium to ensure water quality criteria are met under the 2017 Reasonable Assurance Update (RA Update).  The fifth RA annual assessment report for the 2017-2021 period will be submitted in April 2022.
\end{block}

\begin{block}{2021 Chl-a Monthly Variation Compared to 1974-2020}
\footnotesize
Chlorophyll-a concentrations were evaluated within the bay on a monthly basis during 2021 and compared to prior years' levels (Figure \ref{fig:chlboxplot}). Elevated concentrations in Old Tampa Bay were primarily due to \textit{Pyrodinium bahamense} during the late summer months.  Additionally, unanticipated release of legacy fertilizer process water and mixed seawater from the Piney Point facility in the spring introduced an estimated 205 tons of nitrogen to Lower Tampa Bay.  A subsequent red tide event occurred in Middle and Lower Tampa Bay later in the summer.
\end{block}



\vspace{-0.2in}

\begin{figure}
\centerline{\includegraphics[trim = 0cm 0cm 0cm 0cm, width=1\linewidth]{figure/chlboxplot.pdf}}
\caption{\footnotesize Chlorophyll-a monthly averages from 1975-2020 for the four bay segments. The monthly averages for 2021 are shown in red.}
\label{fig:chlboxplot}
\end{figure}

\vspace{-0.4in}

\begin{block}{Tampa Bay Seagrass Recovery}
\vspace{-0.15in}
\begin{minipage}{0.6\textwidth}
\footnotesize
2020 results showed that Tampa Bay's seagrass coverage fell below the 40,000 acre recovery goal defined in the \href{https://tbep.org/habitat-master-plan-update/}{Habitat Master Plan Update}. The 2020 baywide estimate was 34,298 acres, representing a decrease of 6,354 acres from 2018 (Figure \ref{fig:sgtrnd}). Large decreases were observed in Old Tampa Bay, especially in the Feather Sound area. Increases in the attached algae \textit{Caulerpa prolifera} have also been noted in this region and elsewhere. Research and management plans are currently being developed to address these losses. More information on the bay's seagrass trends using transect monitoring data can be found at \href{https://shiny.tbep.org/seagrasstransect-dash/}{https://shiny.tbep.org/seagrasstransect-dash/} and using the coverage estimates from SWFWMD can be found at \href{https://shiny.tbep.org/seagrass-analysis/}{https://shiny.tbep.org/seagrass-analysis/}.
\end{minipage}
\hspace{0.2in}
\begin{minipage}{0.32\textwidth}
\vspace{0.1in}
\begin{figure}
\includegraphics[width=\textwidth, trim = 0cm 0cm 0cm -1cm]{www/seagrasscov.png}
\caption{\footnotesize Seagrass estimates from 1950-2020 (Source: TBEP \& SWFWMD)}
\label{fig:sgtrnd}
\end{figure}
\end{minipage}
\end{block}

\vspace{-0.15in}

\tiny \textit{\textbf{Note}: 2021 nutrient management compliance assessment available from Sherwood, E., Burke, M., Beck, M.W. 2022. TBEP Technical Report \#xx-22.  Please cite this document as Beck, M.W., Burke, M., Raulerson, G. 2022. 2021 Tampa Bay Water Quality Assessment. TBEP Technical Report \#xx-22, St. Petersburg, FL.} \\

\end{column}

\end{columns}

\end{frame}

\end{document}
